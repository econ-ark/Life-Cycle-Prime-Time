% Created with jtex v.1.0.20
\documentclass{article}
\usepackage{arxiv}

\usepackage[utf8]{inputenc} % allow utf-8 input
\usepackage[T1]{fontenc}    % use 8-bit T1 fonts
\usepackage{hyperref}       % hyperlinks
\usepackage{url}            % simple URL typesetting
\usepackage{datetime}       % show dates in the title block
\usepackage{booktabs}       % professional-quality tables
\usepackage{amsfonts}       % blackboard math symbols
\usepackage{nicefrac}       % compact symbols for 1/2, etc.
\usepackage{microtype}      % microtypography
\usepackage{graphicx}
\usepackage{natbib}
\usepackage{doi}
\usepackage{xcolor}



\hypersetup{colorlinks = true,
linkcolor = purple,
urlcolor  = blue,
citecolor = cyan,
anchorcolor = black}

\title{Literature Appendix}

\newdate{articleDate}{28}{9}{2025}
\date{\displaydate{articleDate}}

\makeatletter
\let\@fnsymbol\@arabic
\makeatother

\author{Christopher Carroll\footnotemark[1]\\
Johns Hopkins University\\Econ-ARK\\\AND
Alan Lujan\\
Johns Hopkins University\\Econ-ARK\\\AND
Matthew N. White\\
Econ-ARK\\}

% Uncomment to override  the `A preprint' in the header
\renewcommand{\headeright}{}
\renewcommand{\undertitle}{}
\renewcommand{\shorttitle}{}

%% Add PDF metadata to help others organize their library
%% Once the PDF is generated, you can check the metadata with
%% $ pdfinfo template.pdf
\hypersetup{
pdftitle={\@title},
pdfsubject={},
pdfauthor={\@author},
pdfkeywords={},
addtopdfcreator={Written in Curvenote}
}

\begin{document}
\maketitle
\footnotetext[1]{Correspondence to: ccarroll@llorracc.org}


\keywords{}

\bigskip\noindent
\begin{tabular}{p{\dimexpr 0.500\linewidth-2\tabcolsep}p{\dimexpr 0.500\linewidth-2\tabcolsep}}
\toprule
Citation & Summary \\
\hline
\cite{Samuelson_1975} & Reviews optimal consumption-investment decisions for investors with constant relative risk aversion utility functions. Shows that optimal portfolio decisions are independent of time, wealth, and consumption. Demonstrates that investing over multiple periods does not increase risk tolerance. \\
\cite{Merton_1969} & Examines optimal portfolio selection and consumption rules in a continuous-time model with stochastic asset returns. Develops optimality equations for multi-asset problems with Wiener processes, with detailed analysis of two-asset cases under constant relative and absolute risk aversion. \\
\cite{Samuelson_1989} & Challenges conventional wisdom that young investors should take more risk than older investors. Shows that when retirement requires minimum wealth levels, investors become less risk-tolerant in youth - the Merton paradox. However, declining human capital with age can reverse this effect. \\
\cite{Epstein_1989} & Develops recursive utility preferences that separate risk attitudes from intertemporal substitution. Shows how these preferences lead to an asset pricing model combining both CAPM and consumption CAPM, where systematic risk depends on both market portfolio returns and consumption growth. \\
\cite{Deaton_1991} & Analyzes saving behavior under borrowing constraints. Shows that with impatient consumers and i.i.d. income, assets act as buffer stock against income shocks. Model can explain microeconomic saving patterns but struggles to match aggregate data unless accounting for heterogeneous income processes. \\
\cite{Bodie_1992} & Examines how labor-leisure choice affects portfolio and consumption decisions over the lifecycle. Shows that labor flexibility allows greater risk-taking in investment portfolios since labor supply can adjust to offset investment losses. \\
\cite{Kimball_1991} & Introduces ``standard risk aversion'' concept where risks that make wealth reductions more painful also make independent risks more painful. Shows this requires both decreasing absolute risk aversion and decreasing absolute prudence. \\
\cite{Jagannathan_1996} & Evaluates common financial planning advice to reduce stock allocation with age. Finds only the human capital argument valid - younger investors should hold more stocks only if their labor income is uncorrelated with stock returns. \\
\cite{Carroll_1997} & Argues that household saving is better explained by ``buffer-stock'' model than traditional lifecycle hypothesis. Shows how income uncertainty and impatience lead consumers to maintain target wealth-to-income ratios, explaining several empirical puzzles. \\
\cite{Viceira_2001} & Shows that employed investors should optimally invest more in stocks than retirees when labor income is uncorrelated with stock returns. Increased labor income risk reduces willingness to hold stocks, with savings being more responsive than portfolio allocation. \\
\cite{Campbell_1999} & Presents analytical solution for portfolio choice with time-varying equity premium. Shows hedging motives can double stock demand for risk-averse investors. Market timing and hedging provide significant welfare benefits. \\
\cite{Barberis_2000} & Examines how return predictability affects long-horizon portfolio choice, accounting for parameter uncertainty. Finds investors should still increase stock allocation with horizon, though less than if ignoring estimation risk. \\
\cite{Madrian_2001} & Studies impact of automatic 401(k) enrollment on savings behavior. Shows significantly higher participation under automatic enrollment, with many participants retaining default contribution rates and allocations due to inertia and perceived implicit advice. \\
\cite{Viceira_2001} & Demonstrates that retirement affects optimal portfolio choice through labor income risk. Employed investors should hold more stocks than retirees when income is uncorrelated with returns, but correlation can reverse this relationship. \\
\cite{Gomes_2002} & Studies lifecycle portfolio choice with constraints and housing costs. Finds stock allocation never exceeds 45\% during working life. Model explains some puzzles but struggles with low market participation rates. \\
\cite{Curcuru_2010} & Reviews evidence on portfolio differences across households. Shows background risk, costs, constraints and lifecycle factors explain some but not all variation. Fixed costs needed to explain non-participation by young/poor households. \\
\cite{GOMES_2005} & Shows lifecycle model with realistic income risk can match participation rates and allocations. Key features are Epstein-Zin preferences, entry costs, and heterogeneous risk aversion. \\
\cite{Cocco_2005} & Solves lifecycle model with non-tradable labor income and borrowing constraints. Shows optimal stock share decreases with age as labor income acts as bond substitute. Quantifies importance of human capital for investment. \\
\cite{Poterba_2006} & Examines how predictability affects optimal long-horizon portfolios. Finds lifecycle strategies similar to constant allocation with same average equity share. Strategy rankings very sensitive to risk aversion, expected returns, and non-401(k) wealth. \\
\cite{Viceira_2007} & Reviews academic models' implications for retirement investment products. Supports risk-based and age-based allocation strategies but suggests improvements to better account for heterogeneity and labor income correlation with returns. \\
\cite{Bodie_2007} & Explores target-date funds as solution for retirement plan participants. Suggests improvements needed for highly risk-averse investors with high market risk exposure through labor income. \\
\cite{Gomes_2008} & Studies lifecycle portfolio choice with flexible labor supply. Shows variable labor significantly raises optimal stock holdings. Finds small welfare costs for well-designed lifecycle funds but large costs for other default options. \\
\cite{Edwards_2008} & Shows self-perceived health risk reduces financial risk-taking after retirement. Effect strongest for singles, suggesting health shocks impact home production. Spouses and bequests provide hedging. \\
\cite{Bhandari_2008} & Finds most pension plan members misunderstand asset allocation. Wealthy self-directed investors show better understanding. Young, educated, high-earning males with planners hold more equity. \\
\cite{Beshears_2009} & Reviews evidence on how defaults impact retirement savings outcomes. Shows defaults significantly affect participation, contribution rates, asset allocation and distributions despite minimal economic barriers to opt-out. \\
\cite{Smetters_2010} & Examines optimal portfolio allocation between stocks and bonds over lifetime. Shows traditional lifecycle models fail to match empirical ``hump'' shape allocation and income-based differences. Demonstrates that including social security benefits correlated with stock returns helps explain observed patterns. \\
\cite{Mitchell_2012} & Studies how target-date funds and choice architecture affect retirement investment decisions. Shows defaults and fund mapping increase participation in employer-selected funds, while demonstrating these funds act as implicit investment advice for active decision-makers. \\
\cite{Guiso_2013} & Reviews evolution and recent developments in household finance field, covering how households use financial markets to achieve their objectives. \\
\cite{Bodie_2015} & Discusses how advances in financial science have enabled improved life-cycle investment products. \\
\cite{Hsu_2015} & Analyzes relative importance of contributions versus investment returns over lifecycle. Shows contributions matter most early on while returns dominate near retirement. Recommends encouraging early contributions and avoiding excessive early equity risk. \\
\cite{O_Hara_2015} & Makes case for target-date funds that evolve only until retirement date rather than beyond, based on human capital depletion and peak retirement risk at retirement start. Provides framework for exploring lifecycle investing questions. \\
\cite{Tang_2015} & Evaluates target-date fund efficiency by calculating welfare loss from suboptimal portfolios. Finds inappropriate glide paths are major source of inefficiency, recommends risk-based fund selection strategy. \\
\cite{Singh_2023} & Examines target-date fund growth and implications. Shows regulatory changes boosted adoption as default options. Analyzes how these funds affect portfolio allocations and investment outcomes. \\
\cite{FAGERENG_2017} & Documents Norwegian household portfolio adjustments with age, showing both rebalancing away from stocks near retirement and market exit after retirement. Estimates model parameters matching this behavior. \\
\cite{Michaelides_2017} & Shows optimal investors should pursue aggressive market timing with time-varying equity premiums. Finds conventional age-based allocation advice is correct on average but ignoring market conditions causes welfare losses. \\
\cite{Blanchett_2018} & Analyzes drivers of target-date fund performance. Finds equity exposure explains only 25\% of return variation, with style and selection decisions accounting for 30\% and 45\% respectively. Suggests evaluating TDF risk requires looking beyond equity weights. \\
\cite{Mulvey_2020} & Develops agent-based simulation framework for retirement planning incorporating multiple asset categories and dynamic allocation strategies. Demonstrates application for setting realistic retirement goals and evaluating policy-rule investment strategies. \\
\cite{Estrada_2020} & Examines why target-date funds become more conservative near retirement despite being suboptimal for capital accumulation. Shows investors would need to double risk aversion in final 25 working years to choose similar allocations. \\
\cite{Hou_2020} & Ranks financial risks facing retirees. Finds longevity and health risks objectively largest, but retirees perceive market risk as highest due to overestimating volatility while underestimating lifespan and health costs. \\
\cite{2014} & Surveys household finance literature covering asset allocation, insurance, trading, retirement saving, mortgages, credit, social factors, financial literacy, and advice. \\
\cite{Gomes_2020} & Reviews lifecycle portfolio choice models incorporating human capital, housing, constraints, background risk and other key factors affecting savings and investment decisions. \\
\cite{Shoven_2021} & Evaluates target-date funds 2010-2020. Finds lower-cost funds match benchmarks while higher-cost deviate. Past performance weakly predicts future. Even near-term funds had significant losses in 2020 crash. \\
\cite{Gomes_2022} & Proposes modified target-date funds exploiting variance risk premium predictability. Shows simplified tactical allocation generates large welfare gains even with constraints and uncertainty. \\
\cite{Duarte_2021} & Uses machine learning to solve complex lifecycle model. Finds substantial welfare losses from simple age-based rules like target-date funds, suggesting benefits from customization by wealth, business cycle, and market valuation. \\
\cite{Mitchell_2021} & Documents massive growth in target-date funds, driven by regulation. Shows adoption increased equity and bond exposure while reducing cash/stock holdings and idiosyncratic risk. Low-cost funds may significantly enhance retirement wealth. \\
\cite{Gomes_2021} & Comprehensive survey of household finance literature covering asset allocation, insurance, trading, retirement saving, mortgages, credit markets, social factors, and financial advice. \\
\cite{Parker_2022} & Documents lifecycle portfolio allocation changes. Shows increased equity exposure early in life and decreased approaching retirement, accelerated by target-date fund adoption after 2006 pension law changes. \\
\cite{Altig_2023} & Analyzes Social Security claiming strategies. Finds most Americans claim too early - over 90\% should wait until 70. Optimal timing could increase lifetime spending by 10-26\% for many households. \\
\cite{Daga_2023} & Proposes lifecycle model incorporating multiple investor goals under uncertain returns and longevity. Uses goal fulfillment gap to quantify strategy value and illustrate optimal decisions. \\
\cite{PARKER_2023} & Shows target-date funds implement contrarian strategies through rebalancing, stabilizing fund flows and dampening stock returns. Growth in these products may reduce market volatility and increase cross-asset correlation. \\
\bottomrule
\end{tabular}

\bigskip



\bibliographystyle{unsrtnat}
\bibliography{main.bib}

\end{document}

% Created with jtex v.1.0.20
\documentclass{article}
\usepackage{arxiv}

\usepackage[utf8]{inputenc} % allow utf-8 input
\usepackage[T1]{fontenc}    % use 8-bit T1 fonts
\usepackage{hyperref}       % hyperlinks
\usepackage{url}            % simple URL typesetting
\usepackage{datetime}       % show dates in the title block
\usepackage{booktabs}       % professional-quality tables
\usepackage{amsfonts}       % blackboard math symbols
\usepackage{nicefrac}       % compact symbols for 1/2, etc.
\usepackage{microtype}      % microtypography
\usepackage{graphicx}
\usepackage{natbib}
\usepackage{doi}
\usepackage{xcolor}

%%%%%%%%%%%%%%%%%%%%%%%%%%%%%%%%%%%%%%%%%%%%%%%%%%
%%%%%%%%%%%%%%%%%%%%  imports  %%%%%%%%%%%%%%%%%%%
\usepackage{amsmath}
%%%%%%%%%%%%%%%%%%%%%%%%%%%%%%%%%%%%%%%%%%%%%%%%%%


\hypersetup{colorlinks = true,
linkcolor = purple,
urlcolor  = blue,
citecolor = cyan,
anchorcolor = black}

\title{Literature Appendix}

\newdate{articleDate}{21}{11}{2024}
\date{\displaydate{articleDate}}

\makeatletter
\let\@fnsymbol\@arabic
\makeatother

\author{Christopher Carroll\footnotemark[1]\\
Johns Hopkins University\\Econ-ARK\\\AND
Alan Lujan\\
Johns Hopkins University AAP\\Econ-ARK\\}

% Uncomment to override  the `A preprint' in the header
\renewcommand{\headeright}{}
\renewcommand{\undertitle}{}
\renewcommand{\shorttitle}{}

%% Add PDF metadata to help others organize their library
%% Once the PDF is generated, you can check the metadata with
%% $ pdfinfo template.pdf
\hypersetup{
pdftitle={\@title},
pdfsubject={},
pdfauthor={\@author},
pdfkeywords={},
addtopdfcreator={Written in Curvenote}
}

\begin{document}
\maketitle
\footnotetext[1]{Correspondence to: ccarroll@llorracc.org}


\keywords{}

\bigskip\noindent
\begin{tabular}{p{\dimexpr 0.500\linewidth-2\tabcolsep}p{\dimexpr 0.500\linewidth-2\tabcolsep}}
\toprule
Citation & Abstract \\
\hline
\cite{Samuelson_1975} & Publisher Summary   This chapter reviews the optimal consumption-investment problem for an investor whose utility for consumption over time is a discounted sum of single-period utilities, with the latter being constant over time and exhibiting constant relative risk aversion (power-law functions or logarithmic functions). It presents a generalization of Phelps' model to include portfolio choice and consumption. The explicit form of the optimal solution is derived for the special case of utility functions having constant relative risk aversion. The optimal portfolio decision is independent of time, wealth, and the consumption decision at each stage. Most analyses of portfolio selection, whether they are of the Markowitzâ€``Tobin mean-variance or of more general type, maximize over one period. The chapter only discusses special and easy cases that suffice to illustrate the general principles involved and presents the lifetime model that reveals that investing for many periods does not itself introduce extra tolerance for riskiness at early or any stages of life. \\
\cite{Merton_1969} & OST models of portfolio selection have M been one-period models. I examine the combined problem of optimal portfolio selection and consumption rules for an individual in a continuous-time model whzere his income is generated by returns on assets and these returns or instantaneous ``growth rates'' are stochastic. P. A. Samuelson has developed a similar model in discrete-time for more general probability distributions in a companion paper [8]. I derive the optimality equations for a multiasset problem when the rate of returns are generated by a Wiener Brownian-motion process. A particular case examined in detail is the two-asset model with constant relative riskaversion or iso-elastic marginal utility. An explicit solution is also found for the case of constant absolute risk-aversion. The general technique employed can be used to examine a wide class of intertemporal economic problems under uncertainty. In addition to the Samuelson paper [8], there is the multi-period analysis of Tobin [9]. Phelps [6] has a model used to determine the optimal consumption rule for a multi-period example where income is partly generated by an asset with an uncertain return. Mirrless [5] has developed a continuous-time optimal consumption model of the neoclassical type with technical progress a random variable. \\
\cite{Merton_1971} &  \\
\cite{Samuelson_1989} & Maximizing expected utility over a lifetime leads one who has constant relative risk aversion and faces random-walk securities returns to be ``myopic'' and hold the same fraction of portfolio in equities early and late in life--a defiance of folk wisdom and casual introspection. By assuming one needs to assure at retirement a minimum (``subsistence'') level of wealth, the present analysis deduces a pattern of greater risk-taking when young than when old. When a subsistence minimum is needed at every period of life, the rentier paradoxically is least risk tolerant in youth--the Robert C. Merton paradox that traces to the decline with age of the present discounted value of the subsistence-consumption requirements. Conversely, the decline with age of capitalized human capital reverses the Merton effect. \\
\cite{Epstein_1989} & This paper develops a class of recursive, but not necessarily expected utility, preferences over intertemporal consumption lotteries. An important feature of these general preferences is that they permit risk attitudes to be disentangled from the degree of intertemporal substitutability. Moreover, in an infinite horizon, representative-agent context, these preference specifications lead to a model of asset returns in which appropriate versions of both the atemporal CAPM and the intertemporal consumption CAPM are nested as special cases. In the authors' general model, systematic risk of an asset is determined by covariance with both the return to the market portfolio and consumption growth. Copyright 1989 by The Econometric Society. \\
\cite{Deaton_1991} & This paper is concerned with the theory of saving when consumers are not permitted to borrow, and with the ability of such a theory to account for some of the stylized facts of saving behavior. When consumers are relatively impatient, and when labor income is independently and identically distributed over time, assets act like a buffer stock, protecting consumption against bad draws of income. The precautionary demand for saving interacts with the borrowing constraints to provide a motive for holding assets. If the income process is positively autocorrelated, but stationary, assets are still used to buffer consumption, but do so less effectively, and at a greater cost in terms of foregone consumption. In the limit, when labor income is a random walk, it is optimal for impatient liquidity constrained consumers simply to consume their incomes. As a consequence, a liquidity constrained representative agent cannot generate aggregate U.S. saving behavior if that agent receives aggregate labor income. Either there is no saving, when income is a random walk, or saving is contracyclical over the business cycle, when income changes are positively autocorrelated. However, in reality, microeconomic income processes do not resemble their average, and it is possible to construct a model of microeconomic saving under liquidity constraints which, at the aggregate level, reproduces many of the stylized facts in the actual data. While it is clear that many households are not liquidity constrained, and do not behave as described here, the models presented in the paper seem to account for important aspects of reality that are not explained by traditional life-cycle models. \\
\cite{Bodie_1992} & Abstract   This paper examines the effect of the labor-leisure choice on portfolio and consumption decisions over an individual's life cycle. The model incorporates the fact that individuals may have considerable flexibility in varying their work effort (including their choice of when to retire). Given this flexibility, the individual simultaneously determines optimal levels of current consumption, labor effort, and an optimal financial investment strategy at each point in his life cycle. We show that labor and investment choices are intimately related. The ability to vary labor supply ex post induces the individual to assume greater risks in his investment portfolio ex ante. \\
\cite{Kimball_1991} & This paper introduces the concept of standard risk aversion. A von Neumann-Morgenstern utility function has standard risk  if any risk makes a small reduction in wealth more painful (in the sense of an increased reduction in expected utility) also makes any undesirable, independent risk more painful. It is shown that, given monotonicity and concavity, the combination of decreasing absolute risk  and decreasing absolute prudence is necessary and sufficient for standard risk aversion. Standard risk  is shown to imply not only Pratt and Zeckhauser's 'proper risk aversion (individually undesirable, independent risks always being jointly undesirable) , but also that being forced to face an undesirable risk reduces the optimal investment in a risky security with and independent return. Similar results are established for the effect of broad class of increases in one risk on the desirability of (or optimal investment in) a second, independent risk. \\
\cite{Jagannathan_1996} & Financial planners typically advise people to shift investments away from stocks and toward bonds as they age. The planners commonly justify this advice in three ways. They argue that stocks are less risky over a young personâ€\texttrademark s long investment horizon, that stocks are often necessary for young people to meet large financial obligations (like college tuition for their children), and that younger people have more years of labor income ahead with which to recover from the potential losses associated with stock ownership. This article uses economic reasoning to evaluate these three different justifications. It finds that the first two arguments do not make economic sense. The last argument is validâ€''but only for people with labor income that is relatively uncorrelated with stock returns. If a personâ€\texttrademark s labor income is highly correlated with stock returns, then that investor is better off shifting investments toward stocks over time. \\
\cite{Carroll_1997} & This paper argues that the typical householdâ€\texttrademark s saving is better described by a “bufferstock” version than by the traditional version of the Life Cycle/Permanent Income Hypothesis (LC/PIH) model. Buffer-stock behavior emerges if consumers with important income uncertainty are sufficiently impatient. In the traditional model, consumption growth is determined solely by tastes; in contrast, buffer-stock consumers set average consumption growth equal to average labor income growth, regardless of tastes. The model can explain three empirical puzzles: the “consumption/income parallel” of Carroll and Summers [1991]; the “consumption/income divergence” first documented in the 1930's; and the temporal stability of the household age/wealth profile despite the unpredictability of idiosyncratic wealth changes. \\
\cite{Viceira_2001} & This paper analyzes optimal portfolio decisions of long-horizon investors with undiversifiable labor income risk and exogenous expected retirement and lifetime horizons. It shows that the fraction of savings optimally invested in stocks is unambiguously larger for employed investors than for retired investors when labor income risk is uncorrelated with stock return risk. This result provides support for the popular recommendation by investment advisors that employed investors should invest in stocks a larger proportion of their savings than retired investors. This paper also examines the effect of increasing labor income risk on savings and portfolio choice and finds that, when labor income risk is independent of stock market risk, a mean-preserving increases in the variance of labor income growth increases the investor's willingness to save and reduce her willingness to hold the risky asset in her portfolio. A sensible calibration of the model shows that savings are relatively more responsive to changes in labor income risk than portfolio demands. Positive correlation between labor income innovations and unexpected asset returns also reduces the investor's willingness to hold the risky asset, because of its poor properties as a hedge against unexpected declines in labor income. This paper also provides intuition on the peculiar form of optimal portfolio choice of very young investors predicted by the standard life-cycle model. \\
\cite{Campbell_1999} & This paper presents an approximate analytical solution to the optimal consumption and portfolio choice problem of an infinitely lived investor with Epstein-Zin-Weil utility who faces a constant riskless interest rate and a time-varying equity premium. When the model is calibrated to U. S. stock market data, it implies that intertemporal hedging motives greatly increase, and may even double, the average demand for stocks by investors whose risk-aversion coefficients exceed one. The optimal portfolio policy also involves timing the stock market. Failure to time or to hedge can cause large welfare losses relative to the optimal policy. \\
\cite{Barberis_2000} & We examine how the evidence of predictability in asset returns affects optimal portfolio choice for investors with long horizons. Particular attention is paid to estimation risk, or uncertainty about the true values of model parameters. We find that even after incorporating parameter uncertainty, there is enough predictability in returns to make investors allocate substantially more to stocks, the longer their horizon. Moreover, the weak statistical significance of the evidence for predictability makes it important to take estimation risk into account; a long-horizon investor who ignores it may overallocate to stocks by a sizeable amount. \\
\cite{Madrian_2001} & This paper analyzes the impact of automatic enrollment on 401(k) savings behavior. We have two key e ndings. First, 401(k) participation is signie cantly higher under automatic enrollment. Second, a substantial fraction of 401(k) participants hired under automatic enrollment retain both the default contribution rate and fund allocation even though few employees hired before automatic enrollment picked this particular outcome. This “ default” behavior appears to result from participant inertia and from employee perceptions of the default as investment advice. These e ndings have implications for the design of 401(k) savings plans as well as for any type of Social Security reform that includes personal accounts over which individuals have control. They also shed light more generally on the importance of both economic and noneconomic (behavioral) factors in the determination of individual savings behavior. \\
\cite{Viceira_2001} & This paper examines how risky labor income and retirement affect optimal portfolio choice. With idiosyncratic labor income risk, the optimal allocation to stocks is unambiguously larger for employed investors than for retired investors, consistent with the typical recommendations of investment advisors. Increasing idiosyncratic labor income risk raises investors' willingness to save and reduces their stock portfolio allocation towards the level of retired investors. Positive correlation between labor income and stock returns has a further negative effect and can actually reduce stockholdings below the level of retired investors. FINANCIAL ADVISORS TYPICALLY RECOMMEND that their customers invest more in stocks than in safe assets when they are working, and to shift their investments towards safe assets when they retire (Jagannathan and Kocherlakota (1996), Malkiel (1996)). By contrast, the standard academic models of portfolio choice (Merton (1969, 1971), Samuelson (1969)) show that retirement is irrelevant for portfolio decisions if investment opportunities are constant and human capital is tradable. In this case the fraction of wealth optimally invested in risky assets should be constant over the lifetime of an individual. Recent research, building on the pioneering work by Merton (1971), shows that time-varying investment opportunities result in portfolio rules with an intertemporal hedging component whose magnitude depends on the investment horizon of the investor (Kim and Omberg (1996), Balduzzi and Lynch (1997), Brennan, Schwartz, and Lagnado (1997), Brandt (1999), Campbell and Viceira (1999, 2001), Barberis (2000)). With time variation in investment opportunities, retirement and death play an instrumental role as events that exogenously fix the individual's investment horizon. More importantly, retirement also marks the time at which individuals stop working and thereafter must live off their lifetime savings and (possibly) transfers. From this perspective, retirement matters for portfolio choice \\
\cite{Gomes_2002} & We study life-cycle asset allocation in the presence of liquidity constraints and undiversifiable labor income risk. The model includes three different assets (cash, long-term government bonds and stocks) and it takes into account the life-cycle profile of housing expenditures. With a modest correlation between stock returns and earnings innovations, the mean share of wealth invested in stocks never exceeds 45\% during working-life. Moreover, the combination of uninsurable human capital and borrowing constraints rationalizes the asset allocation puzzle of Canner, Mankiw and Weil (1997). Nevertheless we argue that asset allocation models must match another important feature of the data: a low stock market participation rate. Along this dimension the model provides a very modest improvement, still predicting a counterfactually high participation rate. We show that this arises from the link between risk aversion and prudence, implying that explanations for the participation puzzle based on the role of background risk are unlikely to succeed. \\
\cite{Curcuru_2010} & In this paper, we summarize and add to the evidence on the large and systematic differences in portfolio composition across individuals with varying characteristics, and evaluate some of the theories that have been proposed in terms of their ability to account for these differences. Variation in background risk exposure from sources such as labor and entrepreneurial income or real estate holdings, and from factors such as transactions costs, borrowing constraints, restricted pension investments and life cycle considerations â€`` can explain some but not all aspects of the observed cross-sectional variation in portfolio holdings in a traditional utility maximizing framework. In particular, fixed costs and life cycle considerations appear necessary to explain the lack of stock market participation by young and less affluent households. Remaining challenges for quantitative theories include the apparent lack of diversification in some unconstrained individual portfolios, and non-participation in the stock market by some households with significant financial wealth. \\
\cite{GOMES_2005} & We show that a life-cycle model with realistically calibrated uninsurable labor income risk and moderate risk aversion can simultaneously match stock market participation rates and asset allocation decisions conditional on participation. The key ingredients of the model are Epsteinâ€``Zin preferences, a fixed stock market entry cost, and moderate heterogeneity in risk aversion. Households with low risk aversion smooth earnings shocks with a small buffer stock of assets, and consequently most of them (optimally) never invest in equities. Therefore, the marginal stockholders are (endogenously) more risk averse, and as a result they do not invest their portfolios fully in stocks. \\
\cite{Cocco_2005} & This article solves a realistically calibrated life cycle model of consumption and portfolio choice with non-tradable labor income and borrowing constraints. Since labor income substitutes for riskless asset holdings, the optimal share invested in equities is roughly decreasing over life. We compute a measure of the importance of human capital for investment behavior. We find that ignoring labor income generates large utility costs, while the cost of ignoring only its risk is an order of magnitude smaller, except when we allow for a disastrous labor income shock. Moreover, we study the implications of introducing endogenous borrowing constraints in this incomplete-markets setting. Copyright 2005, Oxford University Press. \\
\cite{Poterba_2006} & This paper examines how different asset allocation strategies over the course of a worker's career affect the distribution of retirement wealth and the expected utility of wealth at retirement. It considers both rules that allocate a constant portfolio fraction to various assets at all ages, as well as  rules that vary the mix of portfolio assets as the worker ages. The analysis simulates retirement wealth using asset returns that are drawn from the historical return distribution. The results suggest that the distribution of retirement wealth associated with typical lifecycle investment strategies is similar to that from age-invariant asset allocation strategies that set the equity share of the portfolio equal to the average equity share in the lifecycle strategies. There is substantial variation across workers with different characteristics in the expected utility from following different asset allocation strategies. The expected utility associated with different 401(k) asset allocation strategies, and the ranking of these strategies, is very sensitive to three parameters: the expected return on corporate stock, the worker's relative risk aversion, and the amount of non-401(k) wealth that the worker will have available at retirement. At modest levels of risk aversion, or in the presence of substantial non-401(k) wealth at retirement, the historical pattern of stock and bond returns implies that the expected utility of an all-stock investment allocation rule is greater than that from any of the more conservative strategies. Higher risk aversion or lower expected returns on stocks raise the expected utility of following lifecycle strategies or other strategies that reduce equity exposure throughout the lifetime. \\
\cite{Viceira_2007} & This paper reviews recent advances in academic models of asset allocation for long-term investors, and explores their implications for the design of investment products that help investors save for retirement, particularly life-cycle funds and life-style (or balanced) funds. The paper argues that modern portfolio theory provides scientific foundation for the ``risk-based'' asset allocation strategies and the ``age-based'' asset allocation strategies that characterize life-style and life-cycle funds. Risk-based allocation strategies can be optimal in an environment where investors face real interest rate (or reinvestment risk), while human wealth considerations give rise to horizon effects in asset allocation. However, this theory also makes a number of suggestions about how life-style and life-cycle funds should be structured, and shows for which types of investors these funds are appropriate investment choices. Thus, modern portfolio theory provides only qualified support for these funds. Nevertheless, the paper argues that properly designed life-cycle funds are better default investment choices than money market funds in defined-contribution pension plans. The paper also argues for the creation of life-cycle funds that allow for heterogeneity in risk tolerance, and for the creation of life-cycle funds specific to defined-contribution plans that can better account for the correlation between human capital and stock returns. It also suggests that investors who expect to receive Social Security benefits and pension income after retirement should choose a target retirement date for their funds based on their life-expectancy, not their expected retirement date. \\
\cite{Bodie_2007} & Target-date funds (TDFs) for retirement, also known as life-cycle funds, are being offered as a simple solution to the investment task of participants in self-directed retirement plans. A TDF is a “fund of funds” diversified across stocks, bonds, and cash with the feature that the proportion invested in stocks is automatically reduced as time passes. Empirical evidence suggests that a simple TDF strategy would be an improvement over the choices currently made by many uninformed plan participants. This article explores a way to achieve even greater improvement for people who are very risk averse and have high exposure to market risk through their labor. \\
\cite{Gomes_2008} & We investigate optimal consumption, asset accumulation and portfolio decisions in a  realistically calibrated life-cycle model with flexible labor supply. Our framework allows for wage rate uncertainly, variable labor supply, social security benefits and portfolio choice over safe bonds and risky equities.  Our analysis reinforces prior findings that equities are the preferred asset for young households, with the optimal share of equities generally declining prior to retirement. However, variable labor materially alters pre-retirement portfolio choice by significantly raising optimal equity holdings. Using this model, we also investigate the welfare costs of constraining portfolio allocations over the life cycle to mimic popular default investment choices in defined-contribution pension plans, such as stable value funds, balanced funds, and life-cycle (or target date) funds. We find that life-cycle funds designed to match the risk tolerance and investment horizon of investors have small welfare costs. All other choices, including life-cycle funds which do not match investors' risk tolerance, can have substantial welfare costs. \\
\cite{Edwards_2008} & This article investigates the role of self-perceived risky health in explaining continued reductions in financial risk taking after retirement. If future adverse health shocks threaten to increase the marginal utility of consumption, either by absorbing wealth or by changing the utility function, then health risk should prompt individuals to lower their exposure to financial risk. I examine individual-level data from the Study of Assets and Health Dynamics Among the Oldest Old (AHEAD), which reveal that risky health prompts safer investment. Elderly singles respond the most to health risk, consistent with a negative cross partial deriving from health shocks that impede home production. Spouses and planned bequests provide some degree of hedging. Risky health may explain 20\%\% of the age-related decline in financial risk taking after retirement. \\
\cite{Bhandari_2008} & Most defined contribution pension plan members misunderstand asset allocation, but those with higher levels of wealth managing their own money are less likely to be confused. Younger, more-educated, higher-earning advice-receiving males with a planner mindset hold more equity. Notably, an understanding of asset allocation accentuates the impact of the key factors age, income and a planner mindset. \\
\cite{Beshears_2009} & This paper summarizes the empirical evidence on how defaults impact retirement savings outcomes. After outlining the salient features of the various sources of retirement income in the U.S., the paper presents the empirical evidence on how defaults impact retirement savings outcomes at all stages of the savings lifecycle, including savings plan participation, savings rates, asset allocation, and post-retirement savings distributions. The paper then discusses why defaults have such a tremendous impact on savings outcomes. The paper concludes with a discussion of the role of public policy towards retirement saving when defaults matter. John Beshears Department of Economics Harvard University Littauer Center Cambridge, MA 02138 \href{mailto:beshears@fas.harvard.edu}{beshears@fas.harvard.edu} James J. Choi Yale School of Management 135 Prospect Street P.O. Box 208200 New Haven, CT 06520-8200 \href{mailto:james.choi@yale.edu}{james.choi@yale.edu} David Laibson Department of Economics Harvard University Littauer Center Cambridge, MA 02138 \href{mailto:dlaibson@harvard.edu}{dlaibson@harvard.edu} Brigitte C. Madrian Kennedy School of Government Harvard University 79 JFK Street Cambridge, MA 02138 \href{mailto:Brigitte\_Madrian@harvard.edu}{Brigitte\_Madrian@harvard.edu} If transaction costs are small, standard economic theory would suggest that defaults should have little impact on economic outcomes. Agents with well-defined preferences will opt out of any default that does not maximize their utility, regardless of the nature of the default. In practice, however, defaults can have quite sizeable effects on economic outcomes. Recent research has highlighted the important role that defaults play in a wide range of settings: organ donation decisions (Johnson and Goldstein 2003, Abadie and Gay 2004), car insurance plan choices (Johnson et al. 1993), car option purchases (Park, Jun, and McInnis 2000), and consent to receive e-mail marketing (Johnson, Bellman, and Lohse 2003). This paper summarizes the empirical evidence on defaults in another economically important domain: savings outcomes. The evidence strongly suggests that defaults impact savings outcomes at every step along the way. To understand how defaults affect retirement savings outcomes, one must first understand the relevant institutions. Because the empirical literature on how defaults shape retirement savings outcomes focuses mostly on the United States, we begin by describing the different types of retirement income institutions in the United States and some of their salient characteristics. We then present empirical evidence from the United States and other countries, including Chile, Mexico and Sweden, on how defaults influence retirement savings outcomes at all stages of the savings lifecycle, including savings plan participation, savings rates, asset allocation, and post-retirement savings distributions. Next we examine why defaults have such a tremendous impact on savings outcomes. And finally, we consider the role of public policy towards retirement saving when defaults matter. I. Retirement income institutions in the United States There are four primary sources of retirement income for individuals in the United States: (1) social security payments from the government, (2) traditional employer-sponsored definedbenefit pension plans, (3) employer-sponsored defined-contribution savings plans, and (4) individual savings accounts that are tied neither to the government nor to private employers. We will briefly describe each of these institutions in turn. See the Employee Benefit Research Institute (2005) for a more detailed discussion of the U.S. retirement income system. The social security system in the United States provides retirement income to qualified workers and their spouses. While employed, workers and their firms make mandatory contributions to the social security system. Individuals are eligible to claim benefits when they reach age 62, although benefit amounts are higher if individuals postpone their receipt until a later age. Individuals must proactively enroll to begin receiving social security benefits, and most individuals do so no later than age 65. The level of benefits is primarily determined by either an individualâ€\texttrademark s own or his or her spouseâ€\texttrademark s earnings history, with higher earnings corresponding to greater monthly benefit amounts according to a progressive benefits formula. Benefits are also indexed to the cost of living and tend to increase over time because of this. They are paid until an individual dies, with a reduced benefit going to a surviving spouse until his or her death. On average, social security replaces about 40 percent of pre-retirement income, although this varies widely across individuals. Replacement rates tend to be negatively related to income due to the progressive structure of the benefits formula. Benefits are largely funded on a pay-asyou-go basis, with the contributions of workers and firms made today going to pay the benefits of currently retired individuals who worked and paid contributions in the past. There is no private account component to the U.S. social security system, although this is something that has received a great deal of discussion in recent years. Traditionally the second largest component of retirement income has come from employer-sponsored defined-benefit pension plans. These plans share many similarities with the social security system. Benefits are determined by a formula, usually linked to a workerâ€\texttrademark s compensation, age, and tenure. Benefits are usually paid out as a life annuity, or in the case of married individuals as a joint-and-survivor annuity, although workers do have some flexibility in selecting the type of annuity or in opting instead for a lump sum payout. Because traditional defined-benefit pension plans are costly for employers to administer and because they impose funding risk on employers, there has been a movement over the past two decades away from traditional pensions and towards defined-contribution savings plans. There are now more than twice as many active participants in employer-sponsored definedcontribution savings plans as in defined-benefit pension plans, with total assets in defined contribution plans exceeding those in defined benefit plans by more than 10 percent (U.S. Department of Labor 2005). These defined-contribution savings plans come in several different varieties. The most common one, the 401(k), is named after the section of the U.S. tax code that regulates these types of plans. The typical defined-contribution savings plan allows employees to make elective pre-tax contributions to an account over which the employee retains investment control. Many employers also provide matching contributions up to a certain level of employee contributions. The retirement income ultimately derived by the retirees depends on how much they elected to save while working, how generous the employer match was, and the performance of their selected investment portfolios. At retirement, benefits are usually paid in the form of a lump-sum distribution, although some employers offer the option of purchasing an annuity. Relative to traditional defined-benefit pension plans, defined-contribution savings plans impose substantially more risk on individuals while reducing the risks faced by employers. The final significant source of retirement income comes from personal savings accounts that are not tied to an employer (or the government). There are many different ways that individuals can save on their own for retirement, but one particular vehicle, the IRA (for Individual Retirement Account), is very popular because it receives favorable tax treatment. After IRAs were first created, the primary source of funding came from direct individual contributions. Over time, however, restrictions have been placed on the ability of higher-income individuals to make direct tax-favored contributions, and the primary source of IRA funding has shifted to rolloversâ€''transfers of assets from a former employerâ€\texttrademark s defined-contribution savings plan into an IRA. In general, individuals employed at a firm with a defined-contribution savings plan that has an employer match would find that savings plan more attractive than directly contributing to an IRA. Direct IRA contributions largely come from individuals whose employers do not sponsor a defined-contribution savings plan, individuals who are not eligible for their employerâ€\texttrademark s savings plan, or individuals who are not working. The relatively low social security replacement rate (compared to other developed countries) in conjunction with the recent shift towards defined-contribution savings plans and IRAs in the United States has spurred much of the research interest into how defaults and other plan design parameters affect savings outcomes. With individuals bearing greater responsibility for ensuring their own retirement income security, understanding how to improve their savings outcomes has become an important issue both for individuals themselves and for society at large. II. The impact of defaults on retirement savings outcomes: Empirical evidence We now turn to the evidence on how defaults affect retirement savings outcomes, discussing first the effect of institutionally specified defaults, then ‘electiveâ€\texttrademark  defaultsâ€'' mechanisms that are not a pure default, but that share similar characteristics with the institutionally chosen defaults, in terms both of their structure and of their outcomes. A. Savings plan participation In a defined-contribution savings environment, savings plansâ€''whether they are employer-sponsored, government-sponsored, or privately sponsoredâ€''are only a useful tool to the extent that employees actually participate. Recent research suggests that when it comes to savings plan participation, the key behavioral question is not whether or not individuals participate in a savings plan, but rather how long it takes before they actually sign up. The most compelling evidence on the impact of defaults on savings outcomes comes from changes in the default participation status of employees at firms \\
\cite{Smetters_2010} & This paper examines how households should optimally allocate their portfolio choices between risky stocks and risk-free bonds over their lifetime. Traditional lifecycle models in previous work suggest that the allocation toward stocks should start high (near 100\%) early in life and decline over a personâ€\texttrademark s age as human capital depreciates. These models also suggest that, with homothetic utility, the allocation should be roughly independent of a householdâ€\texttrademark s permanent income. The actual empirical evidence, however, indicates more of a “hump” shape allocation over the lifecycle; the lifetime poor also hold a smaller percentage of their portfolio in stocks relative to higher income groups. Households, therefore, appear to be making considerable “mistakes” in their portfolio allocation. Target date funds, which have grown enormously during the past five years, aim to simplify the investment process in a manner consistent with the predictions of this traditional model. We reconsider the portfolio choice allocation in a computationally-demanding lifecycle model in which households face uninsurable wage shocks, uncertain lifetime as well as a progressive and wage-indexed social security system. Social security benefits, therefore, are correlated with stock returns at a low frequency that is more relevant for lifecycle retirement planning. We show that this model is able to more closely replicate the key stylized facts of portfolio choice. In fact, when calibrated to the age-based income-wealth ratios found in the Survey of Consumer Finances, we demonstrate that the portfolio allocation “mistakes” being made by the vast majority of households actually lead to larger levels of welfare relative to the traditional advice incorporated in target date funds. \\
\cite{Mitchell_2012} & Individual responsibility for portfolio construction is a central theme for defined contribution pensions, yet the rise of target-date funds is shifting investment decisions from workers back to employers. A complex choice architecture including automatic enrollment, reenrollment, and fund mapping, is increasing the number of participants defaulting into employer-selected target-date funds. At the same time, portfolios of non-defaulted participants undergo sizeable changes, with equity share ratios widening by over 40 percent points between younger/older participants. Among active decision-makers, these funds act as a form of implicit employer-provided lifecycle investment advice. More broadly, our findings highlight malleable preferences among retirement investors and a demand for default-based guidance or simplified advice for households facing complex choices. \\
\cite{Guiso_2013} & Abstract   Household financeâ€''the normative and positive study of how households use financial markets to achieve their objectivesâ€''has gained a lot of attention over the past decade and has become a field with its own identity, style, and agenda. In this chapter we review its evolution and most recent developments. \\
\cite{Bodie_2015} & Advances in financial science have made possible an improved menu of life-cycle investment products. Reprinted from Financial Analysts Journal, vol. 59, no. 1 (January/February 2003): 24â€``29. Author affiliation is accurate as of the original publication date. \\
\cite{Hsu_2015} & Both investor contributions and investment returns determine retirement plan outcomes, but they have distinctive effects over the investorâ€\texttrademark s lifecycle. Focusing on target-date funds (TDFs) in 401(k) plans, our research demonstrates that in the early stage, contributions are the primary determinant of portfolio balances, whereas allocations have very little impact. In the later stage, returnsâ€''and therefore asset-allocation decisionsâ€''are the principal driver of ending portfolio values. These findings imply that defined contribution plan sponsors should encourage young workers to start contributing to their 401(k) accounts early, consistently, and at a high level. In addition, we recommend seeking TDFs that do not initially take on excessive equity risk, because a heavy loss might have lasting adverse effects on young workersâ€\texttrademark  attitudes toward investing, and equity-heavy allocations tend to have higher fees. We further suggest that plan sponsorsâ€\texttrademark  asset-allocation decisions center on the years near retirement, when returns have the greatest impact. \\
\cite{O_Hara_2015} & “To vs. through” has become shorthand for whether a target-date fund glide path evolves only until the target date or continues to reduce the risk level beyond the target date. This article presents a persuasive common sense case for the to approach, based on an understanding of human capital, which is depleted at retirement, and retirement risk, which is at its highest level the day retirement begins. New research adds empirical force to the argument that a flat post-retirement glide path is superior to a glide path that continues to de-risk beyond the retirement date. The articleâ€\texttrademark s research attempts to create a unified framework for exploring a wide range of lifecycle investing questions and is ultimately able to produce recommendations regarding equity levels, savings rates, and retirement withdrawal strategies. It builds on a substantial body of academic work and seeks to incorporate investor preferences, validated against real-world income and spending data. \\
\cite{Tang_2015} & To evaluate the efficiency of target-date funds (TDFs), one of the fastest growing lines of mutual funds, we take 36 TDF series offered in the market and calculate investorsâ€\texttrademark  welfare loss from TDF investment. We divide welfare loss into two categories: loss from suboptimal risky portfolio and loss from inappropriate glide path. We find that inappropriate glide path constitutes the major source of TDF performance inefficiency. This inefficiency could reduce an investorâ€\texttrademark s annual consumption by up to 17 per cent. We also find that the substantial heterogeneity in TDF performance is primarily caused by variation in glide paths. The heterogeneity contradicts the notion that one TDF fits everyone and it confirms the urgency to match TDF selection to investorsâ€\texttrademark  risk profiles. In this spirit, we advocate a risk-based selection strategy as a remedy for TDF inefficiency. We estimate that this strategy could reduce approximately half of the welfare loss suffered from other commonly used strategies. \\
\cite{Singh_2023} & 1.IntroductionThe Department of Labor's safe harbor provision of 2007 has provided a significant boost to the popularity of target date mutual funds (TDFs). These funds are now included as a default investment option in defined contribution plans along with managed accounts and balanced funds. According to Department of Labor, the Qualified Default Investment Alternative (QDIA) provides a plan sponsor ``safe harbor relief from fiduciary liability for investment outcomes EBSA (2008).'' A consequence of the Pension Protection Act of 2006, employers are increasingly adopting the automatic enrollment option in 401(K) plans, which further supplements the assets under management (AUM) base of TDFs.Investors held about $1 trillion in target date and lifestyle funds at the end of 2014, compared to a total of$2.75 billion at the end of 1995 (ICI Handbook, 2015). This calculates to a compounded annual growth rate (CAGR) of about 35\% over the 10-year period, an impressive flow of investor money to these funds.1A TDF is managed from its purchase date through an expected retirement year, and in some cases, the fund provides an additional ``during retirement'' investment option. On the other hand, a lifestyle fund is directed to a broad age group and tailored to its purported risk tolerance. For example, a lifestyle fund focused on younger investors in their 30s will usually have a high equity exposure, while a lifestyle fund aimed at retirees will likely have a heavy emphasis on fixed income securities. Assuming that one's risk tolerance decreases with age, an investor will move from one lifestyle fund to the next one catering to their ``lifestyle'' as they age, until they are in retirement. In the case of a TDF, this asset allocation is automatically shifted for the investor. A good way to understand the difference between target date and lifestyle funds is to think of lifestyle funds as building blocks of a TDF. So, which of these two fund types should investors prefer? Chang et al. (2014) use a utility maximization framework and use bootstrap simulations to compare welfare benefits of both types of funds. The primary focus of their research is to measure utility derived from fixed and decreasing equity allocation for an individual investor over time. They find that a decreasing equity allocation provides better welfare benefits than a static one. This implies that TDFs are superior investment vehicles from a utility maximization perspective. The authors caution that there is no one-size-fits all TDF, the investor should select such funds based on their risk tolerance. For the purposes of this paper, the term ``target date fund'' is used to designate a broad class of open-end mutual funds and exchange traded funds (ETFs) that include both target date and lifestyle funds.The appeal of a target-date mutual fund lies in the convenience it provides to the investor. She does not have to monitor, and periodically alter the asset allocation because of passage of time. In most cases, the fund mandate automatically provides for that. Based on the chosen retirement age, the fund manager allocates the funds to a predetermined allocation schedule; say 85\% stocks and 15\% fixed income initially when the investor is young. This allocation may eventually reverse to 15\% stocks and 85\% fixed income closer to the ``target date.'' This change in asset allocation that occurs over many years is commonly referred to as the fund's "glide path."The Vanguard Target Retirement Fund 2055 (Ticker: VFFVX), is provided as an example. The 2055 in the fund name indicates the anticipated retirement year for the investor. In 2015, this fund is recommended for a 30-year old anticipating retirement at age 70 or a 25-year old anticipating retirement at age 65. Based on the fund's glide path, the initial asset allocation is quite aggressive, with up to 90\% in equity securities. According to Vanguard, this becomes "more conservative over time, meaning that the percentage of assets allocated to stocks will decrease while the percentage of assets allocated to bonds and other fixed income investments will increase. … \\
\cite{FAGERENG_2017} & Using error-free data on life-cycle portfolio allocations of a large sample of Norwegian households, we document a double adjustment as households age: a rebalancing of the portfolio composition away from stocks as they approach retirement and stock market exit after retirement. When structurally estimating an extended life-cycle model, the parameter combination that best fits the data is one with a relatively large risk aversion, a small per-period participation cost, and a yearly probability of a large stock market loss in line with the frequency of stock market crashes in Norway. \\
\cite{Michaelides_2017} & We solve for optimal consumption and portfolio choice in a life-cycle model with short-sales and borrowing constraints; undiversifiable labor income risk; and a predictable, time-varying, equity premium and show that the investor pursues aggressive market timing strategies. Importantly, in the presence of stock market predictability, the model suggests that the conventional financial advice of reducing stock market exposure as retirement approaches is correct on average, but ignoring changing market information can lead to substantial welfare losses. Therefore, enhanced target-date funds (ETDFs) that condition on expected equity premia increase welfare relative to target-date funds (TDFs). Out-of-sample analysis supports these conclusions. \\
\cite{Blanchett_2018} & The authors explore the relative importance of the three primary drivers of target-date fund (TDF) performance: equity (or market) exposures (which, across a seriesâ€\texttrademark  vintages, combine to form a glide path), style exposures (intrastock and intrabond allocations), and other investment selection decisions (e.g., manager selection and the active/passive decision, as well as any other residual risk exposures). They find that overall equity exposure drives only about 25\% of the variation in returns across TDFs versus approximately 30\% for style and 45\% for selection, on average. Consequently, the analysis of the riskiness of a given TDF must be based on more than the overall weight given to equites. \\
\cite{Mulvey_2020} & The vulnerability of individuals planning for retirement has been growing as a result of the conversion from defined-benefit plans to defined-contribution plans, the steady increase in life longevity, and the uncertainty of asset returns under an ever-changing global environment. A serious problem is the lack of appropriate planning for retirement. How much should an individual in the United States save beyond the Social Security tax to maintain a reasonable lifestyle after retirement? The article designs a framework to facilitate the process of setting realistic goals for retirement planning, featuring the concept of agent-based simulations. Focusing on policy-rule-based investment strategies, the simulation framework includes multiple investable asset categories and explores dynamic allocation based on the investorâ€\texttrademark s age, current salary, and Social Security accumulation situation. Empirical results demonstrate a stylized application of the planning framework. TOPICS: Long-term/retirement investing, pension funds, portfolio management, retirement, Social Security Key Findings • Demonstrates the advantages of agent-based simulation models for addressing the survivability of pension plans. • Compares dynamic and adaptive strategies for integrating saving and investment decisions. • Provides a clear example of goal-based investing for individuals and for institutions such as the US Social Security System. \\
\cite{Estrada_2020} & Target-date funds feature asset allocations that become increasingly conservative as investors approach retirement. An important shortcoming of this strategy is that it is suboptimal in terms of capital accumulation, which begs the question of why these funds are so popular. A possible answer is that investors become more risk averse as they age, gradually favoring more downside protection as they approach retirement. The main issue explored in this article is how much more risk averse would investors need to become during their working years to select asset allocations similar to those in target-date funds; the evidence here shows that investors would have to roughly double their risk aversion during the last 25 years of their working period. An intuitive interpretation of this result, based on how much an individual would pay to avoid a gamble, is also discussed.  TOPICS:Equity portfolio management, retirement, risk management, wealth management  Key Findings  • This article explores why target-date funds switch from a goal of capital accumulation early in the accumulation period to the goal of downside protection as individuals approach retirement.  • This switch is motivated by the increase in risk aversion observed as individuals age, which leads them to value downside protection more as they approach retirement.  • The evidence here shows that if individuals roughly double their risk aversion during the last 25 years of their working period, they would choose asset allocations similar to those featured by the glidepaths of target-date funds. \\
\cite{Hou_2020} & Retirees with limited financial resources face numerous risks, including out-living their money (longevity risk), investment losses (market risk), unexpected health expenses (health risk), the unforeseen needs of family members (family risk), and even retirement benefit cuts (policy risk). This study systematically values and ranks the financial impacts of these risks from both the objective and subjective perspectives and then compares them to show the gaps between retireesâ€\texttrademark  actual risks and their perceptions of the risks in a unified framework. It finds that 1) under the empirical analysis, the greatest risk is longevity risk, followed by health risk; 2) under the subjective analysis, retirees perceive market risk as the highest-ranking risk due to their exaggeration of market volatility; and 3) the longevity risk and health risk are valued less in the subjective ranking than in the objective ranking, because retirees underestimate their life spans and their health costs in late life. \\
\cite{Hou_2020} & Retirees with limited financial resources face numerous risks, including out-living their money (longevity risk), investment losses (market risk), unexpected health expenses (health risk), the unforeseen needs of family members (family risk), and even retirement benefit cuts (policy risk). This study systematically values and ranks the financial impacts of these risks from both the objective and subjective perspectives and then compares them to show the gaps between retireesâ€\texttrademark  actual risks and their perceptions of the risks in a unified framework. It finds that 1) under the empirical analysis, the greatest risk is longevity risk, followed by health risk; 2) under the subjective analysis, retirees perceive market risk as the highest-ranking risk due to their exaggeration of market volatility; and 3) the longevity risk and health risk are valued less in the subjective ranking than in the objective ranking, because retirees underestimate their life spans and their health costs in late life. \\
\cite{2014} & Household financial decisions are complex, interdependent, and heterogeneous, and central to the functioning of the financial system. We present an overview of the rapidly expanding literature on household finance (with some important exceptions) and suggest directions for future research. We begin with the theory and empirics of asset market participation and asset allocation over the life cycle. We then discuss household choices in insurance markets, trading behavior, decisions on retirement saving, and financial choices by retirees. We survey research on liabilities, including mortgage choice, refinancing, and default, and household behavior in unsecured credit markets, including credit cards and payday lending. We then connect the household to its social environment, including peer effects, cultural and hereditary factors, intra-household financial decision-making, financial literacy, cognition, and educational interventions. We also discuss literature on the provision and consumption of financial advice. (JEL D15, G41, G50, J26, Z13) \\
\cite{Gomes_2020} & Life-cycle portfolio choice models capture the role of human capital, housing, borrowing constraints, background risk, and several other crucial ingredients for determining the savings and investme... \\
\cite{Shoven_2021} & This article presents a thorough evaluation of target date funds for the period 2010â€``2020. Target date funds have grown enormously in assets, reaching \$1.4 trillion at the end of 2019, and account for approximately 24\% of all assets in 401(k) accounts. We report on the results of a style analysis evaluation of TDFs that determines their effective asset allocation. It examines both the constant in the style analysis regressions and resulting Sharpe ratios, which reflect the over- or under-performance of the funds relative to a passive benchmark with the same asset allocation. Lower cost TDFs tend to match the benchmark returns, while higher cost TDFs deviate from them considerably. We examine how TDFs performed in the stock market crash between February 19 and March 23, 2020, during which five-week period broad market averages fell by about one-third. We find that the value of long-dated TDFs (those with a target date of 2045 and beyond) fell by between 30\% and 35\%, while the 2025 funds, designed for people roughly 60 years old, lost between 20\% and 25\% of their value. We find that past performance only weakly influences future expected performance. As with equity funds in general in this period, TDFs with actively managed ingredient funds, on average, trailed the performance of their cheaper passively managed counterparts.  TOPICS:Pension funds, portfolio theory, risk management, performance measurement  Key Findings  â--ª Even near term TDFs have considerable equity exposure. For instance, 2025 TDFs lost between 20 and 25\% of their value in the five weeks between February 19 and March 23, 2020. Many longer horizon TDFs did no better than pure equity funds in this period.  â--ª 75\% of actively managed TDFs failed to do as well as the best fitting set of reference ETFs.  â--ª Past performance is only a weak predictor of future performance for TDFs. An extra 1\% per year return in 2010â€``14 period only increases the expected return in 2015â€``19 by 9 basis points per year. \\
\cite{Gomes_2022} & We propose target date funds modified to exploit stock return predictability driven by the variance risk premium. The portfolio rule of these tactical target date funds (TTDFs) is extremely simplified relative to the optimal one, making it easy to implement and to communicate to investors. We show that saving for retirement in TTDFs generates economically large welfare gains, even after we introduce turnover restrictions and transaction costs, and after taking into account parameter uncertainty. This predictability also appears to be uncorrelated with individual household risk, suggesting that households are in a prime position to exploit it. This paper was accepted by Tomasz Piskorski, finance. \\
\cite{Duarte_2021} & We develop a machine-learning solution algorithm to solve for optimal portfolio choice in a lifecycle model that includes many features of reality modelled only separately in previous work. We use the quantitative model to evaluate the consumption-equivalent welfare losses from using simple rules for portfolio allocation across stocks, bonds, and liquid accounts instead of the optimal portfolio choices, both for optimizing households and for households that under-save. We find that the consumption-equivalent losses from using an age-dependent rule as embedded in current target-date/lifecycle funds (TDFs) are substantial, around 2 to 3 percent of consumption, despite the fact that TDF rules mimic average optimal behavior by age closely until shortly before retirement. Optimal equity shares have substantial heterogeneity, particularly by wealth level, state of the business cycle, and dividend-price ratio, implying substantial gains to further customization of advice or TDFs in these dimensions. \\
\cite{Mitchell_2021} & Abstract Target-date funds in corporate retirement plans grew from $5 billion in 2000 to$734 billion in 2018, partly because federal regulation sanctioned these as default investments in automatic enrollment plans. We show that adopters delegated pension investment decisions to fund managers selected by plan sponsors. Inclusion of these funds in retirement saving menus raised equity shares, boosted bond exposures, curtailed cash/company stock holdings, and reduced idiosyncratic risk. The adoption of low-cost target-date funds may enhance retirement wealth by as much as 50\% over a 30-year horizon. \\
\cite{Gomes_2021} & Household financial decisions are complex, interdependent, and heterogeneous, and central to the functioning of the financial system. We present an overview of the rapidly expanding literature on household finance (with some important exceptions) and suggest directions for future research. We begin with the theory and empirics of asset market participation and asset allocation over the life cycle. We then discuss household choices in insurance markets, trading behavior, decisions on retirement saving, and financial choices by retirees. We survey research on liabilities, including mortgage choice, refinancing, and default, and household behavior in unsecured credit markets, including credit cards and payday lending. We then connect the household to its social environment, including peer effects, cultural and hereditary factors, intra-household financial decision-making, financial literacy, cognition, and educational interventions. We also discuss literature on the provision and consumption of financial advice. (JEL D15, G41, G50, J26, Z13) \\
\cite{Parker_2022} & This paper documents the share of investable wealth that middle-class U.S. investors hold in the stock market over their working lives. This share rises modestly early in life and falls significantly as people approach retirement. Prior to 2000, the average investor held less of their investable wealth in the stock market and did not adjust this share over their working life. These changes in portfolio allocation were accelerated by the Pension Protection Act (PPA) of 2006, which allowed employers to adopt target date funds (TDFs) as default options in retirement saving plans. Young retail investors who start at an employer shortly after it adopts TDFs have higher equity shares than those who start at that same employer shortly before the change in defaults. Older investors rebalance more to safe assets. We also study retirement contribution rates over the lifecycle and find that average retirement saving rates increase steadily over the working life. In contrast to what we find for investment in the stock market, contribution rates have been stable over time and across cohorts and were not increased by the PPA. JEL codes: D14, E21, G11, G23; G28; G51 \\
\cite{Altig_2023} & Americans are notoriously bad savers. Large numbers are reaching old age too poor to finance retirements that could last longer than they worked. This study uses the 2018 American Community Survey to impute retirement ages for 2019 Survey of Consumer Finances (SCF) respondents. Next, we run the SCF respondents through the Fiscal Analyzer (TFA) to measure the size and distribution of forgone lifetime Social Security benefits. TFA is a life-cycle, consumption-smoothing research tool that incorporates Social Security and all other major federal and state tax and benefit policies. The program can optimize lifetime Social Security choices. We find that virtually all American workers age 45â€``62 should wait beyond age 65 to collect. More than 90\% should wait till age 70. Only 10.2\% appear to do so. The median loss for this age group in the present value of household lifetime discretionary spending is \$182,370. Optimizing would produce a 10.4\% increase in typical workersâ€\texttrademark  lifetime spending. For one in four, the lifetime spending gain exceeds 17\%. For one in 10, the gain exceeds 26\%. Among the poorest fifth of 45- to 62-year-olds, the median lifetime spending increase is 15.9\%, with one in four gaining more than 27.4\%. \\
\cite{Daga_2023} & Investors have multiple goals throughout their lifetime, each requiring complex interconnected decisions about saving, consumption, and asset allocation. Economists have developed a theoretical solution to solve for lifetime retirement income using stochastic dynamic programming. However, practitioner adoption of this approach has been limited. We theorize that this is due to the inability of stochastic dynamic programming to address multiple investor goals due to the inherent computational complexity in the approach. We put forward a life-cycle model incorporating multiple goals that are relevant for retirement investors in the presence of uncertain future asset returns and longevity. Our objective is to maximize investorsâ€\texttrademark  expected utility over their lifetime. We use a goal fulfillment gap as a metric to quantify the value added by an optimized strategy, and to illustrate the optimal consumption and asset-allocation decisions recommended by the model. \\
\cite{PARKER_2023} & Target date funds (TDFs) are designed to provide unsophisticated or inattentive investors with age-appropriate exposures to different asset classes like stocks and bonds. The rise of TDFs has moved a significant share of retirement investors into macro-contrarian strategies that sell stocks after relatively good stock market performance. This rebalancing drives contrarian flows across equity mutual funds held by TDFs, stabilizing their funding, and reduces stock returns for stocks disproportionately held by these funds when stock market returns are relatively high. Continued growth in TDFs and similar investment products may dampen stock market volatility and increase the transmission of shocks across asset classes. This article is protected by copyright. All rights reserved \\
\bottomrule
\end{tabular}

\bigskip



\bibliographystyle{unsrtnat}
\bibliography{main.bib}

\end{document}

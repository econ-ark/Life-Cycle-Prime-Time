
# Synopsis {-}

The retirement planning landscape is evolving rapidly due to shifting demographics, longer life expectancy, and greater market volatility. As individuals live longer, the need for pension plans and retirement funds to provide flexible, rigorous, and sustainable investment advice has become increasingly urgent.

Traditional economic models, however, have struggled to keep pace with the demands of the financial advice industry, particularly in accurately capturing how individuals manage their portfolios and decumulate wealth post-retirement. For example, the "life cycle model" of optimal saving for retirement is familiar to anyone who has taken an introductory economics class. When hiring a financial advisor, clients likely expect the advisor's job is to tailor these life-cycle model choices to their particular circumstances. However, academics and advisors know that most life-cycle models provide unrealistic guidance—suggesting retirees should run their wealth down to zero and live pension-check to pension-check. This example highlights that the significant shortcomings in economic modeling have so far limited fund managers' ability to translate sound economic principles into actionable insights for their clients.

Recent advancements in theory and computation, however, have given economists new tools—structural models—that bridge the gap between theoretical modeling and real-world data. In this report, we argue that these mathematical and computational innovations enable the creation of portfolio models that can offer advice that is not only data-driven but also sensible, transparent, and customizable for individual clients.

This report presents a model that leverages advances in structural economic modeling, developed using the Econ-ARK platform. Rather than relying on ad hoc theoretical assumptions, we utilize **XYZ data** to guide the theoretical structure and parameter values, ensuring greater accuracy and relevance to how people actually behave.

## Key Insights from the Model {-}

- **Dynamic Asset Allocation**: Our model provides an improved estimate of consumer risk preferences. This allows the glidepath to incorporate more realistic growth asset allocations that balance the need for portfolio growth with the requirement of securing income in retirement.
- **Wealth Preservation Post-Retirement**: Using **XYZ** (e.g., the TRP wealth in the utility function?), we precisely estimate the value retirees place on wealth. Unlike traditional models, this approach reflects the reality that retirees tend to preserve wealth longer, resulting in more sustainable retirement plans that align with observed behaviors.

## Practical Applications {-}

This model serves as a proof of concept for applying cutting-edge structural economic modeling to financial advice. The open-source tools presented with this report can generate customized scenarios based on participant preferences and be adapted to new fund and survey data.

The model tools developed here also allow for ongoing refinement. The modelling can be enriched with more granular investment strategies and client preferences, and ultimately integrated into existing financial workflows to accommodate factors like risk tolerance, investment goals, and demographics.

+++

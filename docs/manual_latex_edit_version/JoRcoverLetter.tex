
       \documentclass[11pt,pdftex,letterpaper]{article}
            \usepackage{setspace}
            \usepackage[dvips,]{graphicx} %draft option suppresses graphics dvi display
%            \usepackage{lscape}
%            \usepackage{latexsym}
%            \usepackage{endnotes}
%            \usepackage{epsfig}
            \usepackage{amsmath}
%           \singlespace
            \setlength{\textwidth}{6.5in}
            \setlength{\textheight}{9in}
            \addtolength{\topmargin}{-\topmargin} 
            \setlength{\oddsidemargin}{0in}
            \setlength{\evensidemargin}{0in}
            \addtolength{\headsep}{-\headsep}
            \addtolength{\topskip}{-\topskip}
            \addtolength{\headheight}{-\headheight}
            \setcounter{secnumdepth}{2}
%            \renewcommand{\thesection}{\arabic{section}}
            % \renewcommand{\footnote}{\endnote}
            \newtheorem{proposition}{Proposition}
            \newtheorem{definition}{Definition}
            \newtheorem{lemma}{lemma}
            \newtheorem{corollary}{Corollary}
            \newtheorem{assumption}{Assumption}
            \newcommand{\Prob}{\operatorname{Prob}}
            \clubpenalty 5000
            \widowpenalty 5000
            \renewcommand{\baselinestretch}{1.38}
            \usepackage{amsmath}
            \usepackage{amsthm}
            \usepackage{amsfonts}
            \usepackage{bbm}
            \usepackage{hyperref}
            \newcommand{\N}{\mathbb{N}}
			\newcommand{\R}{\mathbb{R}}
			\setlength{\parindent}{0pt}
			\setlength{\parskip}{1em}
						
\begin{document}

\begin{singlespace}
	Matthew N. White\\
	4400 Sedgwick Road\\
	Baltimore, MD 21210
	
	\vspace{0.25cm}
	
	December 8, 2025
	
	\vspace{0.25cm}
	
	P.\ Brett Hammond, Editor\\
	Journal of Retirement\\
	41 Madison Ave, 20th floor\\
	New York, NY 10010	
\end{singlespace}

\vspace{0.25cm}

Dr.\ Hammond:

Attached please find a manuscript for submission to the \textit{Journal of Retirement}, entitled ``The Life-Cycle Model is Ready for Prime Time.'' Its thrust is straightforward: Financial planning professionals have shied away from life-cycle consumption-saving models because (under basic assumptions) the framework implies that retirees should ``optimally'' plan to run their wealth down to nearly zero and then consume hand-to-mouth in old age. However, somewhat recent developments with respect to both theory and data have provided new tools for life-cycle models to \textit{not} provide advice wildly at odds with retirees' revealed preferences.

The core of the paper is the presentation of three life-cycle consumption-saving models with portfolio allocation using open source tools, with the basic point that there are several model features that \textit{could} be added to the standard framework that would rectify the ``drawdown failure.'' We further discuss other alternative assumptions or features that could yield similar results, with commentary for their relationship to financial advisers' legal and ethical obligations.

I acknowledge that our paper is somewhat longer than preferred by the \textit{Journal of Retirement}'s submission guidelines. We were torn on which parts to cut--- background discussion that is seemingly inessential might actually be of great interest to your readership! If you and your co-editors are amenable to the paper, we would be open to any feedback about its scope.

Thank you very much for your time, and we look forward to your response.


\vspace{1cm}

\begin{singlespace}
\hspace{3cm}Sincerely,

\vspace{0.5cm}

\hspace{3cm}Matthew N. White
\end{singlespace}

\thispagestyle{empty}
\end{document}